%% margin: The margin between outer edge of the poster and the edge of the paper; i.e. ultimate margin
%% innermargin: The margin from the edge of the poster edge to the outermost edge of the blocks; i.e. content margin
%% colspace (horitzontal collumn space?) needs to be high before it's noticable
% innermargin was 40pt, idk what's printable
\documentclass[25pt, a0paper, portrait, margin=0pt, innermargin=40pt, colspace=40pt, blockverticalspace=45pt]{tikzposter}

% Typography
% disable hypnetation
%% \usepackage[none]{hyphenat} % behaves kinda odd, requires multiple compiles?
\usepackage{textcomp}
\usepackage{hyphenat}
\usepackage[all]{nowidow}
\usepackage{babel}
\usepackage{microtype}

\usepackage{blindtext}
\usepackage{xcolor}
\usepackage{multicol}
\usepackage{fix-cm}
\usepackage{stackengine}
\usepackage[utf8]{inputenc}
\usepackage{enumitem}
\usepackage{geometry}

% figures
\usepackage{graphicx}
\usepackage{float}
\usepackage{caption}
\DeclareCaptionFormat{postersize}{\fontsize{30}{36}\selectfont#1#2#3}
\captionsetup{format=postersize}
\usepackage{tabularx}
%% \usepackage{wrapfig} % doesn't work so far

% tikz
\usepackage{standalone}
\usetikzlibrary{positioning}
\usetikzlibrary{shapes.geometric}
\usetikzlibrary{calc}
\usetikzlibrary{decorations.pathreplacing}
%% \usetikzlibrary{arrows}
\usetikzlibrary{arrows.meta}
\newcommand\vcdots{\;\stackMath\stackinset{c}{0pt}{c}{0.6ex}{\vdots}{\vphantom{-}}\;}

% Table packages
\usepackage{booktabs}

% Math
\usepackage{amssymb}
\usepackage{amsmath}
%% sans serif math font
\usepackage[cmbright]{sfmath}
%% \usepackage[EULERGREEK]{sansmath}
%% \sansmath


% Bibliography
%% \usepackage[ backend=biber, sorting=none, maxcitenames=2, natbib]{biblatex}
%% \addbibresource{poster.bib}
\usepackage[square,numbers, sort&compress]{natbib}
\bibliographystyle{short}
%% \bibliographystyle{unsrtnat}
%% \bibliographystyle{agsm}
% this only redefines the citations in text
\renewcommand{\bibnumfmt}[1]{\textbf{[#1]} \,}
% remove "References" heading in natbib
\renewcommand{\bibsection}{}

% Font settings
%% \usepackage[]{fontspec}
%% \defaultfontfeatures{Mapping=tex-text,Scale=MatchLowercase}
%% \setmainfont{Open Sans}
\usepackage[T1]{fontenc}
%% this scaling feature is pretty handy
\usepackage[default,scale=0.79]{opensans}
\DeclareRobustCommand\ltseries{\fontseries{l}\selectfont}


% based on "Minimal" blockstyle
% body ofsett 1 is nice bc of the titleless innerblock "glitch"
% (see stackexchange link below)
\defineblockstyle{eccv2}{
    titlewidthscale=1, bodywidthscale=1, titlecenter,
    titleoffsetx=0pt, titleoffsety=0pt, bodyoffsetx=0pt, bodyoffsety=0pt,
    bodyverticalshift=-6pt, roundedcorners=6, linewidth=0.43cm, % same as innersep
    titleinnersep=1cm, bodyinnersep=0.21cm
}{
    \begin{scope}[line width=\blocklinewidth, rounded corners=\blockroundedcorners]
      %% uncomment the ifs for different titleless look
       \ifBlockHasTitle %
           % box outline
           \draw[draw=eccvgrayblue, fill=white]
               (blockbody.south west) rectangle (blocktitle.north east);
           % title rectangle
           \fill[eccvgrayblue]
               (blocktitle.south west) rectangle (blocktitle.north east);
        \fi
    \end{scope}
}

% Consider making the line either blue or gray
\defineblockstyle{eccv2optimizations}{
    titlewidthscale=1, bodywidthscale=1, titlecenter,
    titleoffsetx=0pt, titleoffsety=0pt, bodyoffsetx=0pt, bodyoffsety=0pt,
    bodyverticalshift=-6pt, roundedcorners=6, linewidth=0.42cm, % same as innersep
    titleinnersep=1cm, bodyinnersep=0.21cm
}{
    \begin{scope}[line width=\blocklinewidth, rounded corners=\blockroundedcorners]
      % box outline
      \draw[draw=eccvgrayblue, fill=none]
      	(blockbody.south west) rectangle (blocktitle.north east);
      % title rectangle
      \fill[eccvgrayblue]
      	(blocktitle.south west) rectangle (blocktitle.north east);
      % seperator line
      \path (blockbody.south) 
      % sorta add these corrdinates/values to the start
      +(0.0cm,6cm) coordinate (c1);
      \path (blockbody.north)
      +(0.0cm, -1.225cm) coordinate (c2);
      %% \draw [line width=0.42cm,eccvgrayblue, dashed, dash pattern=on 20pt off 20pt] (blockbody.south) -- (blockbody.north);
      \draw [line width=0.15cm,eccvgrayblue] (c1) -- (c2);
    \end{scope}
}

\defineblockstyle{reference}{
    titlewidthscale=1, bodywidthscale=1, titlecenter,
    titleoffsetx=0pt, titleoffsety=0pt, bodyoffsetx=0pt, bodyoffsety=0pt,
    bodyverticalshift=0pt, roundedcorners=0, linewidth=0.0cm, % same as innersep
    titleinnersep=0cm, bodyinnersep=0pt % matches innerbody style
}{}

% the titleoffsety is important, see
% https://tex.stackexchange.com/questions/447378/tikzposter-leaves-a-gap-in-empty-innerblock-title
\defineinnerblockstyle{eccvinner}{
    titlewidthscale=1, bodywidthscale=1, titlecenter,
    titleoffsetx=0pt, titleoffsety=-0pt, bodyoffsetx=0pt, bodyoffsety=0pt,
    bodyverticalshift=0pt, roundedcorners=0, linewidth=0.0cm,
    titleinnersep=0pt, bodyinnersep=25pt
}{
    \begin{scope}[line width=\innerblocklinewidth, rounded corners=\innerblockroundedcorners]
       \ifInnerblockHasTitle %
           \filldraw[innerblockbodybgcolor]
               (innerblockbody.south west) rectangle (innerblocktitle.north east);
       \else
             \filldraw[innerblockbodybgcolor]
                 (innerblockbody.south west) rectangle (innerblockbody.north east);
        \fi
    \end{scope}
}

\defineinnerblockstyle{eccvinneroptimizations}{
    titlewidthscale=1, bodywidthscale=1, titlecenter,
    titleoffsetx=0pt, titleoffsety=-0pt, bodyoffsetx=0pt, bodyoffsety=0pt,
    bodyverticalshift=0pt, roundedcorners=0, linewidth=0.0cm,
    titleinnersep=0pt, bodyinnersep=25pt
}{}

% Redefine block title stuff
\usepackage{xpatch}
\xpatchcmd{\block}{\bf\LARGE\color{blocktitlefgcolor}}{\bfseries\huge\color{eccvblue}}{}{}
\xpatchcmd{\innerblock}{\bf \color{innerblocktitlefgcolor}}{\bfseries\color{eccvblue}}{}{}
% remove space that empty innerblock title leaves behind
\makeatletter
\xpatchcmd{\innerblock}{\node[minimum width=\TP@innerblocktitlewidth, minimum height=\TP@innerblocktitleheight, anchor=center] (innerblocktitle)}{\node[inner sep=0pt, minimum width=\TP@innerblocktitlewidth, minimum height=\TP@innerblocktitleheight, anchor=center] (innerblocktitle)}{}{}
\makeatother

%% redo font stuff here I guess
%% Inner body text style
\xpatchcmd{\block}{#3}{\fontsize{30}{36}#3}{}{}

\definecolor{eccvlightblue}{HTML}{009EE0}
\definecolor{eccvlightblued}{HTML}{6898be}
% more gray blue ish but makes a good background color
% combine with background titles in like light blue?
\definecolor{eccvgrayblue}{HTML}{E2F1F9}
\definecolor{eccvsuperlightblue}{HTML}{F5F9FF}
%% \definecolor{eccvgrayblue}{HTML}{D7FFFF}
\definecolor{eccvblue}{HTML}{045494}
\definecolor{eccvdarkblue}{HTML}{023258}
%% \definecolor{lightgrey}{HTML}{777777}
\definecolor{lightgrey}{HTML}{CCCCCC}

%% define a gray for all the blocks (iterm gray?) and github blue
%% Primary colors are iTerm gray (to be combined with blue backgrounds?)
%% black and eccv blue
\definecolorstyle{eccvblue} {
\definecolor{colorOne}{named}{white}
\definecolor{colorTwo}{named}{eccvlightblue}
\definecolor{colorThree}{named}{eccvblue}
}{
% Background Colors
\colorlet{backgroundcolor}{white}
\colorlet{framecolor}{black}
% Title Colors
\colorlet{titlefgcolor}{colorThree}
\colorlet{titlebgcolor}{colorOne}
% Block Colors
\colorlet{blocktitlebgcolor}{colorThree}
\colorlet{blocktitlefgcolor}{white}
\colorlet{blockbodybgcolor}{white}
\colorlet{blockbodyfgcolor}{black}
% Innerblock Colors
\colorlet{innerblocktitlebgcolor}{white}
\colorlet{innerblocktitlefgcolor}{black}
\colorlet{innerblockbodybgcolor}{colorThree!30!white}
\colorlet{innerblockbodyfgcolor}{black}
% Note colors
\colorlet{notefgcolor}{black}
\colorlet{notebgcolor}{colorTwo!50!white}
\colorlet{noteframecolor}{colorTwo}
}



\definetitlestyle{simpletitle}{
width=807mm, roundedcorners=0, linewidth=0pt, innersep=0pt,
titletotopverticalspace=40pt, titletoblockverticalspace=45pt}{% more precise?
\begin{scope}[line width=\titlelinewidth, rounded corners=\titleroundedcorners]
\draw[color=white, fill=titlebgcolor] % notice the white hack
(\titleposleft,\titleposbottom) rectangle (\titleposright,\titlepostop);
\end{scope}
}

\newcommand\boldblue[1]{\textcolor{eccvblue}{\textbf{#1}}}
\newcommand\markblue[1]{\textcolor{eccvblue}{{#1}}}


\tikzposterlatexaffectionproofoff

% Make the title layout horizontal instead of vertical, as in Tom's Example?
% that way you can squeeze more information in
\title{SUBITIZING WITH VARIATIONAL AUTOENCODERS}
\author{Rijnder Wever, Tom Runia}
\date{\today}
\institute{University of Amsterdam}
\titlegraphic{\includegraphics[width=3.7cm]{UVA-plain}}

\usecolorstyle[colorOne=white,colorTwo=black,colorThree=eccvblue]{eccvblue}
\usetitlestyle{simpletitle}

\settitle{
    %% \begin{multicols}{2}
    %% \setstackgap{L}{1em}
    %% \vspace*{-1\topskip plus 1fill}
  	\noindent\begin{minipage}{0.69\linewidth}
    \vbox{
      \color{titlefgcolor} {\bfseries \fontsize{80}{96} \selectfont \@title \par}
      \vspace*{1.21em}
      % The two different colors in the title are really not that nice
      %% \color{titlefgcolor} {\huge \@author {\ltseries Intelligent Sensory Information Systems}}}
      \color{titlefgcolor} {\huge \@author\  --- Intelligent Sensory Information Systems}}
    %% \vspace*{\fill}
    %% \columnbreak
    \end{minipage}
    \hfill%
    \begin{minipage}{0.28\linewidth}
      \vspace*{-10pt}
      \noindent\begin{minipage}{0.6\textwidth}\raggedleft
        \@titlegraphic
      \end{minipage}
      \hfill%
      \hspace*{-20pt}
      \begin{minipage}{0.4\textwidth}\raggedleft
        \hspace*{-25pt}\huge \color{black}University of\\\vspace{-5mm}Amsterdam
      \end{minipage}
    \end{minipage}

    %% \end{multicols}
}


\begin{document}

%% \useblockstyle{Basic}
\useblockstyle{eccv2}
%% \useinnerblockstyle{Basic}
\useinnerblockstyle{eccvinner}

\maketitle

\block[bodyinnersep=0.00cm]{}{
\colorlet{innerblockbodybgcolor}{white}
% No need for margins if there is no box
% looks a tiny bit weirder without margins?
% make sure the line thickness matches the box thickness
% The rule could possibly be done in tikz
{
  %% \huge
  %% Needs to contain some reference to subitizing
  %% \innerblock[bodyinnersep=40pt]{}{
  \begin{center}
    \begin{minipage}{1.0\linewidth}
      {\Large
      % nog niet erg smooth
      % ook te lang (misschien kan de 1e zin weg of anders)
      % 	- Als je de 1e zin weghaalt kun je misschien bij de laatste zin
      % 	  iets doen als "Provided that..."
        Numerosity, the number of objects in a set, is a basic property of visual scenes. Inspired by \citet{stoianov2012} we propose an \boldblue{unsupervised} generative model to learn visual numerosity representations from natural and synthetic image datasets designed for instance counting within the subitizing range. Specifically, we use a hierarchically organized \boldblue{convolutional variational autoencoder} (VAE) for encoding and reconstructing training images. Provided that numerosity is a key characteristic in the images, the network will learn to encode visual numerosity in the latent representation.\par}
      \end{minipage}
  \end{center}
\vspace*{5pt} % whitespace between title and content
}}
% Horitzontal rule that looks only sorta okay and takes up space
%%   \vspace*{10pt}
%%  \noindent\makebox[\linewidth]{\textcolor{lightgrey}{\rule{0.75\paperwidth}{0.12cm}}}}


\begin{columns}
\column{0.54}

\block{CONTRIBUTIONS}{

\colorlet{innerblockbodybgcolor}{white}
\innerblock[bodyinnersep=25pt, bodyverticalshift=-00pt]{}{
  \begin{itemize}[leftmargin=*, label={\hspace*{-2pt}\huge\raisebox{-0.25ex}{\textcolor{eccvblue}\textbullet}}]
  \setlength\itemsep{0.8em}
  %% \item Our VAE spontaneously develops the ability to encode numerisotiy in the subitizing range from natural images.
  %% \item Training a VAE on natural images results in a representation that encodes numerosity in the subitizing range.
  %% \item Our VAE spontaneously develops the ability to encode numerosity within the subitizing range when trained on natural images.
  \item Our VAE spontaneously encodes visual numerosity in the subitizing range when trained on natural images.
  %% \item Visual numerosity in the subitizing range emerges as a statistical property of natural images in our VAE.
  \item  Numerosity and object area are represented separately, similar to biological neural networks.
  \item A loss function aimed at contour reconstruction improves the network's subitizing performance.
  \item Numerosity information represented in the latent space is successfully extrapolated to a subitizing task.
  \end{itemize}
}
}

\block{VARIATIONAL AUTOENCODER}{
\colorlet{innerblockbodybgcolor}{white}
\innerblock[bodyoffsetx=-40pt]{}{
  \vspace*{20.6pt}
  \hspace*{-14pt}\includestandalone{vae_diagram}
  \vspace*{21.9pt}
}
\colorlet{innerblockbodybgcolor}{eccvgrayblue}
\innerblock{}{
\boldblue{VAEs} are generative models that perform \boldblue{unsupervised} representation learning. 
%Unlike the deterministic latent representation used in conventional autoencoders, VAEs map data samples to a posterior distribution $Q$ representing the latent space. 
The VAE's objective function is the combination of a reconstruction term and a KL regularization:
%
\begin{align}
  %\log P(X) - \mathcal{D}_{KL}[ Q(z \mid X) \mid\mid P(z \mid X)] = E[\log P(X \mid z) - \mathcal{D}_{KL}[ Q(z \mid X) \mid\mid P(z)]]
  \mathcal{L}_{VAE} = E[\log P(X \mid z)] - \mathcal{D}_{KL}[ Q(z \mid X) \mid\mid P(z)]
\end{align}}
}


\column{0.46}

\block{SUBITIZING}{
\colorlet{innerblockbodybgcolor}{white}
% drop "instances" after the "1-4"?

%% \begin{figure}%
%% \subfloat{\includegraphics[width=5.15cm,valign=c]{vstack_e}}
%% \quad
%% \subfloat{\includegraphics[width=6cm,valign=c]{e4}}
%% \end{figure}
\innerblock{}{
\begin{tikzfigure}
%% \includegraphics{vstack_e}
\vspace*{-10.5pt}
\includegraphics[width=0.245\linewidth]{s1} 
\includegraphics[width=0.245\linewidth]{s2} 
\includegraphics[width=0.245\linewidth]{s3} 
\includegraphics[width=0.245\linewidth]{s4r} 
\end{tikzfigure}
%% \colorlet{innerblockbodybgcolor}{white}
}
\colorlet{innerblockbodybgcolor}{eccvgrayblue}
% More point like?
% the subitizing range modulates the cognitive mechanisms directed at instance counting
\innerblock{}{\boldblue{Subitizing} is a perceptual ability that enables rapid and spontaneous identification of the numerosity of small visual sets. When the \boldblue{subitizing range} of 1 -- 4 instances is exceeded, other cognitive mechanisms related to instance counting are invoked. We explore the emergence of visual number sense in unsupervised deep networks trained on \boldblue{natural images} from the \boldblue{Salient Object Subitizing Dataset} \citep{zhang2016salient}. 
% The complexity of natural visual scenes helps capture the abstract nature of number sense.
}
}

\block{VISUAL NUMBER SENSE}{
\colorlet{innerblockbodybgcolor}{white}
\innerblock{}{
%% the one paper (pre-Nieder) with "integral part of the visual world" maybe has figures as well. Reflection of the visiospatial systems... (see paper?) How does the perceptually immediate character of subitizing arise? (visiospatial system)
% "Sudden" character of subitizing
% Cogntive proccesing of indicates , indicating that numerosity is a perceptual category.
% maybe wrapfig works better but couldn't get it to work
\begin{minipage}{0.57\linewidth}
%% \begin{hyphenrules}{nohyphenation}
\vspace*{-17pt}
\begin{itemize}[leftmargin=*, label={\hspace*{-2pt}\huge\raisebox{-0.25ex}{\textcolor{eccvblue}\textbullet}}]
  \setlength\itemsep{1.20em}
  %% \item Separate systems for exact and approximate numerosity representation
  \item \boldblue{Parallel processing} achieves rapid perceptual discrimination of exact numerosity \citep{dehaene2011number}.
   %% \item Automatic map generation  of salient objects is performed in the visiospatial system.
   \item The visiospatial system performs \boldblue{nonmediated} spatial map generation of salient objects \cite{piazza2009humans}.
  %% \item Subitizing is a \boldblue{nonmediated} process originating in the visiospatial system \cite{piazza2009humans} 
  \item Neural populations can automatically develop the ability to judge visual numerosity.
\end{itemize}
%% \end{hyphenrules}
%% Immediate but limited exact \boldblue{visual numerosity} processing abilities could be a result of a capacity-limited multiple object individuation mechanism that performs \boldblue{parallel processing} of salient objects \citep{poncet2016individuation}. The cause of it's \boldblue{sudden} character is hypothesized to be a byproduct of the visiospatial system's automatic salient object map generation of a limited number of salient locations in the visual field \citep{piazza2009humans}. 
\end{minipage}
\begin{minipage}{0.55\linewidth}
%% \vspace*{-6pt}
\begin{tikzfigure}
  \vspace*{-10pt}
  \hspace*{-92pt}\includegraphics[width=0.74\textwidth, height=12.0cm]{subcrop}
\end{tikzfigure}
\end{minipage}
}
}

\end{columns}

%% \begin{columns}
%% \column{0.5}
\useblockstyle{eccv2optimizations}
% Misschien: RECONSTRUCTIONAL OPTIMIZATIONS? 
\block{OPTIMIZING THE RECONSTRUCTIONS}{
%% \begin{minipage}[t]{0.4\linewidth}

\useinnerblockstyle{eccvinneroptimizations}
\colorlet{innerblockbodybgcolor}{white}
%% \renewcommand{\columnseprulecolor}{\color[HTML]{E2F1F9}}
%% \setlength{\parskip}{0pt plus 1pt}
%% \setlength{\oddsidemargin}{0pc}
%% \setlength{\marginparwidth}{0pc}
%% \setlength{\topmargin}{0.0pc}
%% \setlength\columnseprule{0.15cm}

%% \setlength\multicolsep{-1.1pt} % this number is dependend on the fontsize

%% \setcounter{columnbadness}{7000}
%% \setcounter{finalcolumnbadness}{7000}
%% \setcounter{collectmore}{-1}
%% \begin{multicols}{2}
\innerblock[bodyinnersep=25pt]{}{
  \centering
  \begin{minipage}{0.494\linewidth}
    \centering
    \begin{tikzfigure}
      \vspace*{-8pt}
      \hspace*{-9.5pt}\includegraphics[width=0.8\linewidth]{bce_dfc_crop}
    \end{tikzfigure}
  \end{minipage}\hfill
  \begin{minipage}{0.494\linewidth}
    \centering
    %% \hspace*{45pt}
    \begin{tikzfigure}
      \vspace*{-10pt}
      \hspace*{15pt}\includegraphics[width=0.17\linewidth]{syn1}\hspace{3pt}%
      \includegraphics[width=0.17\linewidth]{syn2}\hspace{3pt}%
      \includegraphics[width=0.17\linewidth]{syn3}\hspace{3pt}%
      \includegraphics[width=0.17\linewidth]{syn4}\\%
      %% \bigskip
      \vspace*{3pt}
      \hspace*{15pt}\includegraphics[width=0.17\linewidth]{syn1b}\hspace{3pt}%
      \includegraphics[width=0.17\linewidth]{syn2b}\hspace{3pt}%
      \includegraphics[width=0.17\linewidth]{syn3b}\hspace{3pt}%
      \includegraphics[width=0.17\linewidth]{syn4b}%
    \end{tikzfigure}%
 \end{minipage}%
  
}
\colorlet{innerblockbodybgcolor}{eccvgrayblue}

\useinnerblockstyle{eccvinner}
\innerblock{}{
\renewcommand{\columnseprulecolor}{\color[HTML]{FFFFFF}}
\setlength\columnsep{90pt}
\setlength\columnseprule{0.14cm}
%% \setlength\columnseprule{0.0cm}
\begin{multicols}{2}
% Could drop the first sentence from this block, and the second of the other
{\boldblue{Feature Perceptual Loss} \citep{hou2017deep} improved object contour reconstruction by using intermediate layer representations from a pretrained network in the objective function of the VAE.\par}
{Augmentation of the SOS dataset with \boldblue{synthetic data} was required to familiarize the model with an extensive distribution of object types and the spatial configurations thereof.}
%% \vskip 0.11em
\end{multicols}
}
}

\useblockstyle{eccv2}
\begin{columns}
\column{0.49}


\block{SIZE-INVARIANT NUMEROSITY DETECTORS}{
\colorlet{innerblockbodybgcolor}{white}
\innerblock{}{
  % all tikz figures need slight manual recentering
  % https://tex.stackexchange.com/questions/447615/tikzposter-how-to-center-a-tikzfigure-within-a-block
  \begin{tikzfigure}
    \vspace*{-20pt}
    \includegraphics[width=0.409\linewidth]{NN}\hspace{55pt}
    \includegraphics[width=0.409\linewidth]{NA}
  \end{tikzfigure}
\vspace*{-3pt}
% Maybe Fig.
\captionof*{table}{\boldblue{Figure 1.} Responses of latent dimensions to changes in object area and numerosity.}
}
\colorlet{innerblockbodybgcolor}{eccvgrayblue}
\innerblock{}{
\vspace*{-12.0pt}
\begin{tikzfigure}
  \includegraphics[width=0.392\textwidth]{bird_varied_crop}
\end{tikzfigure}
\vspace*{-1pt}

}
\colorlet{innerblockbodybgcolor}{white}
\innerblock{}{
  Latent dimensions of the learned representations encode complex information. Numerosity is likely encoded \boldblue{invariant of cumulative object area}, as in biological neural networks.
  %% Comparable with biological neural networks, an analysis of the learned representations revealed that numerosity is represented . However, no latent dimension represents just one visual property. %
}
}

\column{0.51}

\block{SUBITIZING TASK PERFORMANCE}{
\colorlet{innerblockbodybgcolor}{white}
\innerblock{}{
\vspace*{95pt}
\begin{center}
  %% Possibly make "avg. classification precs." italic
  % Remove state-of-the-art, there's mostly lower performing algorithms
  \captionof*{table}{\boldblue{Table 1.}  Average classification precision ({\%}) of count labels to images from the SOS test dataset.}
\end{center}
\vspace*{-25pt}
\begin{tikzfigure}
    \begin{tabularx}{0.95\linewidth}{lXXXXXX}
    \toprule
    Count Label $\rightarrow$ & \quad 0 & \quad 1 & \quad 2 & \quad 3 & \quad 4+ & \quad mean \\
    \midrule
    %% \rowcolor
    Chance & \quad 27.5 & \quad46.5 & \quad18.6 & \quad11.7 & \quad\phantom{0}9.7 & \quad22.8 \\
    GIST & \quad 67.4 & \quad65.0 & \quad32.3 & \quad17.5 & \quad24.7 & \quad41.4 \\
    SIFT+IFV & \quad 83.0 & \quad68.1 & \quad35.1 & \quad26.6 & \quad38.1 & \quad50.1 \\
    CNN\_FT & \quad 93.6 & \quad93.8 & \quad75.2 & \quad58.6 & \quad71.6 & \quad78.6 \\
    VAE + softmax (ours) & \quad76.0 & \quad49.0 & \quad40.0 & \quad27.0 & \quad30.0 & \quad44.4 \\
    \bottomrule
    \end{tabularx}
\end{tikzfigure}
\vspace*{95pt}
}
\colorlet{innerblockbodybgcolor}{eccvgrayblue}
\innerblock{}{
\vspace*{20pt}
We compare our \boldblue{unsupervised approach} to existing supervised approaches to instance counting. The strength of the representation learned by the VAE is measured by training a simple \boldblue{softmax classifier} that is fed latent representations of images with corresponding count labels.
\vspace*{20pt}
} % numbers are not math font as well

}

\useblockstyle{reference}
% not sure if this actually centers text but it is more centered I guess
 %% screws with the centering
\renewcommand*{\bibfont}{\normalsize}

\block[bodyinnersep=5pt, bodyoffsetx=19pt]{}{
%% \vspace*{15pt}
\noindent\makebox[\linewidth]{\textcolor{lightgrey}{\rule{0.40\paperwidth}{0.12cm}}}
\vspace*{-39pt}
{
  %% \begin{center}
    %% \begin{minipage}{0.94\linewidth}
    % maybe print as /small text
    {
      \bibliography{poster}
      %% \printbibliography[heading=none]
    }
    %% \end{minipage}
  %% \end{center}
}
}

\end{columns}

%


\end{document}
